\documentclass[a4paper,twoside]{article}

\usepackage{epsfig}
\usepackage{subfigure}
\usepackage{calc}
\usepackage{amssymb}
\usepackage{amstext}
\usepackage{amsmath}
\usepackage{amsthm}
\usepackage{multicol}
\usepackage{pslatex}
\usepackage{apalike}
\usepackage{SCITEPRESS}  % Please add other packages that you may need BEFORE the SCITEPRESS.sty package.

\subfigtopskip=0pt
\subfigcapskip=0pt
\subfigbottomskip=0pt

\begin{document}

\title{Authors' Instructions  \subtitle{Preparation of Doctoral Consortium Contributions} }

\author{\authorname{First Author Name\sup{1}, Second Author Name\sup{1} and Third Author Name\sup{2}}
\affiliation{\sup{1}Institute of Problem Solving, XYZ University, My Street, MyTown, MyCountry}
\affiliation{\sup{2}Department of Computing, Main University, MySecondTown, MyCountry}
\email{\{f\_author, s\_author\}@ips.xyz.edu, t\_author@dc.mu.edu}
}

\onecolumn \maketitle \normalsize \vfill

\section{\uppercase{RESEARCH PROBLEM}}
\label{sec:objectives}

\noindent Your Extended Abstract will be part of the Doctoral Consortium
digital book, therefore we ask that authors follow the
guidelines explained in this example in order to achieve an
uniform compilation of received contributions.

Be advised that Extended Abstracts in a technically unsuitable
form will be returned for retyping. After returned the manuscript
must be appropriately modified.


\section{\uppercase{OUTLINE OF OBJECTIVES}}

\noindent The use of this document in mandatory for the preparation
of the camera-ready. Please follow the instructions closely in order
to make the volume look as uniform as possible \cite{Moore99}.

Please remember that all the contributions must be in English
and without orthographic errors.

Do not add any text to the headers (do not set running heads) and
footers, not even page numbers, because text will be added electronically.

For a best viewing experience the used font must be Times New
Roman, except on special occasions, such as program code 4.3.7.

\section{\uppercase{STATE OF THE ART}}

\noindent For Doctoral Consortium papers authors are required to have at least the following main sections: (\textbf{RESEARCH PROBLEM, OUTLINE OF OBJECTIVES, STATE OF THE ART, METHODOLOGY, EXPECTED OUTCOME, STAGE OF THE RESEARCH}).
All other sections can be added if needed as well as subsections and subsubsections. References are not mandatory.
\vfill

\section{\uppercase{METHODOLOGY}}

\noindent The paper size must be set to A4 (210x297 mm). The document
margins must be the following:

\begin{itemize}
    \item Top: 3,3 cm;
    \item Bottom: 4,2 cm;
    \item Left: 2,6 cm;
    \item Right: 2,6 cm.
\end{itemize}

Any text or material outside the aforementioned margins will not be printed.



\subsection{Manuscript Setup}

\noindent The template is composed by a set of 7 files, in the
following 2 groups:\\
\noindent {\bf Group 1.} To format your paper you will need to copy
into your working directory, but NOT edit, the following 4 files:
\begin{verbatim}
  - apalike.bst
  - apalike.sty
  - article.cls
  - scitepress.sty
\end{verbatim}

\noindent {\bf Group 2.} Additionally, you may wish to copy and edit
the following 3 example files:
\begin{verbatim}
  - example.bib
  - example.tex
  - scitepress.eps
\end{verbatim}


\subsection{First Section}

This section must be in one column and is comprised of the
Extended Abstract title, sub-title (Optional) and authors
names and affiliations.


\subsubsection{Title and Subtitle}

Use the command \textit{$\backslash$title} and follow the given structure in "example.tex". The title and subtitle must be with initial letters
capitalized (titlecased). If no subtitle is required, please remove the corresponding \textit{$\backslash$subtitle} command. In the title or subtitle, words like "is", "or", "then", etc. should not be capitalized unless they are the first word of the subtitle. \textbf{No formulas or special characters of any form or language are} allowed in the title or subtitle.



\subsubsection{Authors and Affiliations}

Use the command \textit{$\backslash$author} and follow the given structure in "example.tex".


\subsection{Second Section}

Files "example.tex" and "example.bib" show how to create a paper
with a corresponding list of references.

This section must be in two columns.

Each column must be 7,5-centimeter wide with a column spacing
of 0,8-centimeter.

The section text must be set to 10-point.

Section, subsection and sub-subsection first paragraph should not
have the first line indent.

To remove the paragraph indentation (only necessary for the
sections), use the command \textit{$\backslash$noindent} before the
paragraph first word.

If you use other style files (.sty) you MUST include them in the
final manuscript zip file.

\subsubsection{Section Titles}

The heading of a section title should be in all-capitals.

Example: \textit{$\backslash$section\{FIRST TITLE\}}


\subsubsection{Subsection Titles}

The heading of a subsection title must be with initial letters
capitalized (titlecased).

Words like "is", "or", "then", etc. should not be capitalized unless
they are the first word of the subsection title.

Example: \textit{$\backslash$subsection\{First Subtitle\}}

\subsubsection{Sub-Subsection Titles}

The heading of a sub subsection title should be with initial letters
capitalized (titlecased).

Words like "is", "or", "then", etc should not be capitalized unless
they are the first word of the sub subsection title.

Example: \textit{$\backslash$subsubsection\{First Subsubtitle\}}

\subsubsection{Tables}

Tables must appear inside the designated margins or they may span
the two columns.

Tables in two columns must be positioned at the top or bottom of the
page within the given margins. To span a table in two columns please add an asterisk (*) to the table \textit{begin} and \textit{end} command.

Example: \textit{$\backslash$begin\{table*\}}

\hspace*{1.5cm}\textit{$\backslash$end\{table*\}}\\

Tables should be centered and should always have a caption
positioned above it. The font size to use is 9-point. No bold or
italic font style should be used.

The final sentence of a caption should end with a period.

\begin{table}[h]
\caption{This caption has one line so it is
centered.}\label{tab:example1} \centering
\begin{tabular}{|c|c|}
  \hline
  Example column 1 & Example column 2 \\
  \hline
  Example text 1 & Example text 2 \\
  \hline
\end{tabular}
\end{table}

\begin{table}[h]
\caption{This caption has more than one line so it has to be
justified.}\label{tab:example2} \centering
\begin{tabular}{|c|c|}
  \hline
  Example column 1 & Example column 2 \\
  \hline
  Example text 1 & Example text 2 \\
  \hline
\end{tabular}
\end{table}

Please note that the word "Table" is spelled out.


\subsubsection{Figures}

Please produce your figures electronically, and integrate them into
your document and zip file.

Check that in line drawings, lines are not interrupted and have a
constant width. Grids and details within the figures must be clearly
readable and may not be written one on top of the other.

Figure resolution should be at least 300 dpi.

Figures must appear inside the designated margins or they may span
the two columns.

Figures in two columns must be positioned at the top or bottom of
the page within the given margins. To span a figure in two columns please add an asterisk (*) to the figure \textit{begin} and \textit{end} command.

Example: \textit{$\backslash$begin\{figure*\}}

\hspace*{1.5cm}\textit{$\backslash$end\{figure*\}}

Figures should be centered and should always have a caption
positioned under it. The font size to use is 9-point. No bold or
italic font style should be used.

\begin{figure}[!h]
  %\vspace{-0.2cm}
  \centering
   {\epsfig{file = SCITEPRESS.eps, width = 5.5cm}}
  \caption{This caption has one line so it is centered.}
  \label{fig:example1}
 \end{figure}

\begin{figure}[!h]
  \vspace{-0.2cm}
  \centering
   {\epsfig{file = SCITEPRESS.eps, width = 5.5cm}}
  \caption{This caption has more than one line so it has to be justified.}
  \label{fig:example2}
  \vspace{-0.1cm}
\end{figure}

The final sentence of a caption should end with a period.

Please note that the word "Figure" is spelled out.
\vfill

\subsubsection{Equations}

Equations should be placed on a separate line, numbered and
centered.\\The numbers accorded to equations should appear in
consecutive order inside each section or within the contribution,
with the number enclosed in brackets and justified to the right,
starting with the number 1.

Example:

\begin{equation}\label{eq1}
    a=b+c
\end{equation}

\subsubsection{Program Code}\label{subsubsec:program_code}

Program listing or program commands in text should be set in
typewriter form such as Courier New.

Example of a Computer Program in Pascal:

\begin{small}
\begin{verbatim}
 Begin
     Writeln('Hello World!!');
 End.
\end{verbatim}
\end{small}


The text must be aligned to the left and in 9-point type.


\subsubsection{Reference Text and Citations}

References and citations should follow the Harvard (Author, date)
System Convention (see the References section in the compiled
manuscript). As example you may consider the citation
\cite{Smith98}. Besides that, all references should be cited in the
text. No numbers with or without brackets should be used to list the
references.

References should be set to 9-point. Citations should be 10-point
font size.

You may check the structure of "example.bib" before constructing the
references.


\section{\uppercase{EXPECTED OUTCOME}}

\noindent Please use this section to describe in detail the proposed method using visual element such as figures and tables if needed. Don't forget that every section can be complemented by the use of subsections and subsubsections.\vfill


\section{\uppercase{STAGE OF THE RESEARCH}}
\noindent This section should describe in detail what the expected outcome of your Phd. Thesis is.

\bibliographystyle{apalike}
{\small
\bibliography{example}}


\section*{\uppercase{Appendix}}

\noindent If any, the appendix should appear directly after the
references without numbering, and not on a new page. To do so please use the following command:
\textit{$\backslash$section*\{APPENDIX\}}


\end{document}

